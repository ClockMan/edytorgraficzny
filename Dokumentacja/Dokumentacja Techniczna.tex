\documentclass[a4paper,10pt]{article}
\usepackage[utf8]{inputenc}
\usepackage[polish]{babel}
\usepackage{polski}
\usepackage{graphicx}
\usepackage{hyperref}
\hypersetup{colorlinks,citecolor=black,filecolor=black,linkcolor=black,urlcolor=black,pdftex}

\title{\textbf{\begin{LARGE}P.I.W.O. 1.0\end{LARGE}}\\\underline{Projekt Informatyczny} \\ Wilqu \& Others \\\textbf{Dokumentacja Techniczna}}

\author{Piotr Wilk \and Piotr Zegar \and Mateusz Tylek \and Mateusz Kocąb \and Wojciech Zbiegieł \and Sławomir Librant \and  Marek Prząda}
\begin{document}
\maketitle
\begin{figure}[h]
 \centering
\includegraphics{logo}
\begin{center}
\begin{center}

\end{center}

 % logo.eps: 0x0 pixel, 300dpi, 0.00x0.00 cm, bb=0 0 160 154
\end{center}

\end{figure}
\newpage
\tableofcontents
\newpage
\section{Etap wstępny}
\subsection{Opis dziedziny przedmiotowej}
Dziedziną projektu jest grafika, w~głównej mierze operację cyfrowe. 
\subsection{Cel projektu – po co?}
Celem projektu jest zrealizowanie programu umożliwiającego cyfrowe przetwarzanie obrazu, jako z wizualizowanego ciągu bloków, na każdym z nich będzie dodana możliwość poglądu obrazu w~każdym jego stadium przekształcania. Użytkownik będzie mógł dodawać własne typy danych jaki i funkcji operujących na nich. 
\subsection{Zakres projektu – co i jak?}
\begin{enumerate}
\item stworzenie dokumentacji
\item zrealizowanie graficznego interfejsu
\item implementacja operacji przetwarzania obrazu takich jak: \\
(wybór pozostawiony dla pozostałych osób z grupy)
\item dodanie opcji tworzenia własnych dodatków (plug-in)
\end{enumerate}
\subsection{Opracowanie wymagań wstępnych}
\subsubsection{Oczekiwana funkcjonalność systemu}
Możliwość:
\begin{enumerate}
 \item tworzenia bloków reprezentujących wybrane operacje cyfrowe.
 \item tworzenie własnych dodatków.
 \item wczytania różnych formatów obrazów.
 \item zapis powstałych obrazów.
 \item zapis aktualnego stanu programu (położenia i połączeń bloków).
\end{enumerate}
\subsubsection{Opis rzeczywistych obiektów i zależności między nimi}

\subsubsection{Ograniczenia (system, środowisko, specyficzne wymagania)}
Program zostanie napisany w IDE Borland C++, darmowa biblioteka FreeImage umożliwi wczytywanie wielu formatów plików.
\subsection{Harmonogram prac}
\begin{enumerate}
 \item Stworzenie dokumentacji projektu - Piotr Wilk
 \item Zaprojektowanie oraz Implementacja silnika aplikacji - Piotr Zegar, Piotr Wilk
 \item Implementacja GUI - Piotr Zegar
 \item Pisanie wtyczek - pozostałe osoby.
\end{enumerate}
\end{document}
