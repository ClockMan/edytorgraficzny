\documentclass[a4paper,10pt]{article}
\usepackage[utf8]{inputenc}
\usepackage[polish]{babel}
\usepackage{polski}
\usepackage{graphicx}
\usepackage{hyperref}
\hypersetup{colorlinks,citecolor=black,filecolor=black,linkcolor=black,urlcolor=black,pdftex}

\title{\textbf{\begin{LARGE}P.I.W.O. 1.0\end{LARGE}}\\\underline{Projekt Informatyczny} \\ Wilqu \& Others \\\textbf{Instrukcja użytkownika}}

\author{Piotr Wilk \and Piotr Zegar \and Mateusz Tylek \and Mateusz Kocąb \and Wojciech Zbiegieł \and Sławomir Librant \and  Marek Prząda}
\begin{document}
\maketitle
\begin{figure}[h]
 \centering
\includegraphics{logo}
\begin{center}
\begin{center}

\end{center}

 % logo.eps: 0x0 pixel, 300dpi, 0.00x0.00 cm, bb=0 0 160 154
\end{center}

\end{figure}
\newpage
\tableofcontents
\newpage

\section{Wstęp}
P.I.W.O 1.0 jest silnikiem napisanym z myślą o cyfrowym przetważaniu obrazów aczkolwiek dzięki dośc rozbudowanej obsłudze pluginów można nim przetważać dowolne dane. W programie cykl operacji jakie mają zostać wykonane na obiekcie definiujemy w postaci bloczków które można ze sobą łączyć.
\section{Wymagania sprzętowe}

\section{Interfejs użytkownika}
Główne okno programu jest podzielone na 4 stefy:
\begin{enumerate}
 \item Menu główne
 \item Toolbar
 \item Strefa projektu
 \item System logów
 \item Pasek status
\end{enumerate}
W tytule okna zawsze zawarta jest nazwa projektu/plik projektu. Gwiazdka na końcu sugeruje iż w projekcie zostały wprowadzone zmiany.
\subsection{Menu główne}
\subsubsection{Plik}
Menu zawierające opcje pozwalające na zarządzanie projektem.
\begin{description}
\item[Nowy projekt] Zamyka aktualnie otwarty projekt. W przypadku gdy zostały w nim wprowadzone zmiany od czasu ostatniego zapisu lub niebył on zapisywany wcześniej użytkownik zostanie zapytany o zapis projektu lub porzucenie zmian. Tworzy nowy projekt.
\item[Otwórz projekt] Zamyka aktualnie otwarty projekt. W przypadku gdy zostały w nim wprowadzone zmiany od czasu ostatniego zapisu lub niebył on zapisywany wcześniej użytkownik zostanie zapytany o zapis projektu lub porzucenie zmian. Otwiera okno pozwalające na wybranie pliku z którego zostanie wczytany nowy projekt.
\item[Zapisz projekt] Opcja dostępna tylko gdy mamy już otwarty projekt. Jeśli nie był on wcześniej zapisywany to zostanie otwarte okno w którym użytkownik będzie mógł wybrac położenie i nazwę pliku pod którą projekt zostanie zapisany. W przeciwnym wypadku program zapisze projekt używając pliku docelowego który wcześniej już został wybrany.
\item[Zapisz projekt jako] Opcja dostępna tylko gdy mamy już otwarty projekt. Zostanie otwarte okno w którym użytkownik będzie mógł wybrac położenie i nazwę pliku pod którą projekt zostanie zapisany.
\item[Zamknij projekt] Opcja dostępna tylko gdy mamy już otwarty projekt. Zamyka aktualnie otwarty projekt. W przypadku gdy zostały w nim wprowadzone zmiany od czasu ostatniego zapisu lub niebył on zapisywany wcześniej użytkownik zostanie zapytany o zapis projektu lub porzucenie zmian.
\item[Zakończ] Zamyka program, jeśli mamy otwarty projekt i przypadku gdy zostały w nim wprowadzone zmiany od czasu ostatniego zapisu lub niebył on zapisywany wcześniej użytkownik zostanie zapytany o zapis projektu lub porzucenie zmian. 
\end{description}
\subsubsection{Edycja}
Wszystkie opcje z tego menu są dostepne tylko i wyłącznie gdy mamy otwarty projekt. Niektóre z nich wymagają aby w projekcie było conajmniej jedno połączenie, blok, lub aby conajmniej jeden blok był zaznaczony lub połączenie.
\begin{description}
\item[Zaznacz wszystkie bloki] Polecenie zaznacza wszystkie bloki. Odznacza aktualnie zaznaczone połączenie jeśli jakieś jest. Jeśli już wszystkie bloki są zaznaczone przez wywołaniem polecenia to niema ono efektu. 
\item[Odznacz wszystkie bloki] Polecenie odznacza wszystkie zaznaczone bloki jeśli jakieś są.
\item[Odwróć zaznaczenie bloków] Polecenie odznacza zaznaczone bloki i zaznacza te które nie są zaznaczone. Odznacza aktualnie zaznaczone połączenie jeśli jakieś jest.
\item[Duplikuj zaznaczone bloki] Polecenie duplikuje aktualnie zaznaczone bloki i wszystkie połączenia pomiedzy tymi blokami.
\item[Usuń wszystkie bloki] Polecenie usuwa wszystkie połączenia i bloki z projektu.
\item[Usuń zaznaczone bloki] Polecenie usuwa wszystkie zaznaczone bloki wraz z połączeniami do nich podłaczonymi.
\item[Odznacz zaznaczone połaczenie] Polecenie odznacza aktualnie zaznaczone połączenie jeśli jest takie.
\item[Usuń wszystkie połączenia] Polecenie usuwa wszystkie połączenia miedzy blokami z projektu.
\item[Usuń zaznaczone połączenie] Polecenie usuwa aktualnei zaznaczone połączenie jeśli jest takie.
\item[Zresetuj wszystkie połaczenia] Polecenie anuluje zmiany wprowadzone przez użytkownika w pozycji
\end{description}

\subsubsection{Historia}
Menu służy do zarządzania oknami historii bloczków. Aktywne tylko w przypadku otwartych okien histori. Posiada 2 statyczne funkcje:
\begin{description}
\item[Zamknij wszystkie okna] Zamyka wszystkie okna historii. 
\item[Pokaż wszystkie okna] Wyciąga wszystkie otwarte okna histori na pierwszy plan.
\end{description}
Poniżej tych 2 pozycji w zależności od ilości otwartych okien będą widniały kolejne opcje po jednej na każde okno pozwalające na wyciągnięcie na wierz dowolne okno histori atualnie otwate.

\subsubsection{Uruchom}
Menu pozwalające wykonać operacje dostarczone przez pluginy. Menu aktywne tylko w przypadku posiadania otwartego projektu z conajmniej jednym bloczkiem.
\begin{description}
\item [Auto uruchamianie] Opcja typu ``toogle'', pozwala na włączanie trybu auto-uruchamiania projektu.
\item [Uruchom wszystko] Opcja uruchamia projekt, przetważane są wszystkie mozliwe bloczki nie uwzględniając ostatnie uruchomienia.
\item [Spradź projekt] Opcja wymusza przeprowadzenie sprawdzenia wszystkich bloczków.
\item [Anuluj] Opcja przerywa uruchamianie projektu.
\end{description}

\subsubsection{Menu dynamicznych bloczków}
W tym miejscu znajdują się menu ładowane z plików koniguracyjnych, każda opcja pozwala na dodanie ściśle powiązanego z nia bloczka, ilośc tych menu i struktura nie jest ściśle określona.

\subsubsection{Pomoc}
\begin{description}
\item [Instrukcja użytkownika] Wyświetla ten plik.
\item [Dokumentacja techniczna] Wyświetla dokument zawierający informacje przydatne dla developerów lub osób które by chiały wprowadzic zmiany do projektu.
\item [O autorach] Wyświetla okno pokazujące informacje o autorach programu jak i ich zanagażowanie.
\item [O programie] Wyświetla okno pokazujące krótkie informacje o programie.
\end{description}

\subsection{Toolbar}
Toolbar jest to szereg przyciskow diocznych zaraz pod menu. Ich funkcje są takie same jak ich odpowiednikom w menu a kolejść jest nastepująca:
\begin{itemize}
 \item Nowy projekt
 \item Otwórz projekt
 \item \textit{Separator}
 \item Zapisz projekt
 \item Zapisz projekt jako
 \item \textit{Separator}
 \item Zamknij projekt
 \item \textit{Separator}
 \item Zaznacz wszystkie bloki
 \item Duplikuj wszystkie bloki
 \item Usuń zaznaczony blok lub połączenie
 \item Zresetuj połączenie
 \item \textit{Separator}
 \item Auto-uruchamianie
 \item Uruchom
 \item Sprawdź projekt
 \item Anuluj
 \item \textit{Separator}
 \item Przycisk pozwalający na dodanie ostatnio dodawanego bloczka
 \item Przycisk pozwalający na dodanie drugiego w kolejności ostatnio dodawanego bloczka
 \item Przycisk pozwalający na dodanie trzeciego w kolejności ostatnio dodawanego bloczka
 \item Przycisk pozwalający na dodanie czwartego w kolejności ostatnio dodawanego bloczka
 \item Przycisk pozwalający na dodanie piątego w kolejności ostatnio dodawanego bloczka
 \item \textit{Separator}
 \item Instrukcja użytkownika
 \item O autorach
 \item O programie
\end{itemize}

\subsection{Strefa projektu}
Przestrzeń w której zarządzamy bloczkami/połaczeniami. Omówimy ją dokładniej później.

\subsection{System logów}
System wyświetlania logów jest podzielony na 3 strefy:
\begin{description}
 \item[Główny log] Wyświetlane są tu wszystkie logi oprócz Debug, Run log
 \item[Run log] Wyświetlane są tu informacje natemat przetważanych bloczków, Logi te są usuwane przed każdym uruchomieniem projektu.
 \item[Debug log] Wyświetlane sa tu wszystkie możliwe logi.
 \end{description}
Kolory komunikatów:
\begin{description}
 \item[Czarny] - informacja
 \item[Niebieski] - debug
 \item[Zólty] - ostrzeżenie
 \item[Czerwony] - błąd
 \item[Zielony] - sukces
 \end{description}
Klikając prawym przyciskiem na liście logów pokaże się menu pozwalające na wyczyszczenia aktualnych logów lub zapisanie ich do pliku.
\subsection{Pasek status}
Aktualnie jedyną funkcja paska statusu jest wyświetlanie aktualnie otwartego projektu.
\section{Skróty klawiatury}
\begin{description}
 \item[Ctrl+N] - Nowy projeky
 \item[Ctrl+O] - Otórz projekt
 \item[Ctrl+S] - Zapisz projekt
 \item[Shift+Ctrl+S] - Zapisz projekt jako
 \item[Ctrl+C] - Zamknij projekt
 \item[Ctrl+X] - Zakończ
 \item[Ctrl+A] - Zaznacz wszystkie bloki
 \item[Ctrl+U] - Odznacz wszystkie bloki
 \item[Ctrl+I] - Odwróć zaznaczenie bloków
 \item[Ctrl+D] - Duplikuj zaznaczone bloki
 \item[Shift+Del] - Usuń wszystkie bloki
 \item[Alt+D] - Usuń zaznaczone bloki
 \item[Ctrl+E] - Odznacz zaznaczone połączenie
 \item[Ctrl+Del] - Usuń wszystkie połączenia
 \item[Shift+D] - Usuń zaznaczone połączenie
 \item[Shift+R] - Zresetuj wszystkie połaczenia
 \item[Ctrl+R] - Zresetuj zaznaczone połączenie
 \item[Del] - Usuń zaznaczone bloki lub połączenie
 \end{description}
\section{Zarządzanie projektem}
\subsection{Bloczki}
\subsubsection{Wejścia}
\subsubsection{Wyjścia}
\subsubsection{Przycisk konfiguracyjny}
\subsubsection{Status}
\subsection{Połączenia}
\subsection{Historia}
\subsection{Eksport/Import}
\subsection{Uruchamianie}
\section{Podsumowane}

\end{document}
